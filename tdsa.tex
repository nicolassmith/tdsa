\documentclass[aps,prd,superscriptaddress,nofootinbib,showpacs,letterpaper]{revtex4-1}

\usepackage{verbatim}
\usepackage[usenames,dvipsnames]{color}
\usepackage[pdftex]{graphicx}
\usepackage[english]{babel}
\usepackage{amsmath,amssymb}
\usepackage[colorlinks=true,citecolor=blue,linkcolor=magenta]{hyperref}
%\special{papersize=8.5in,11in}
\renewcommand{\baselinestretch}{1.3}

%---macros---%
\newcommand{\B}[1]{\textcolor{blue}{#1}} %comment this out later


\begin{document}

\title{It ain't easy being Squeezey}

\author{Some Dude}
\author{Some Ohter Dude}


\begin{abstract}
Squeezing is awesome because it lets us do rad science stuff.  In this paper we
describe that stuff.
\end{abstract}

\maketitle

\section{Introduction}

\section{Frequency dependent squeezing}

\section{Corbitt et al idea}

\section{Time dependent squeezing}

\subsection{Fixed squeezing angle}

\subsection{Following chirps to increase total number of detections}

\subsection{Increasing SNR on the loudest detections}

\section{Comparisons}

\section{Conclusions}

%\bibliography{tdsa.bib}

\end{document}

